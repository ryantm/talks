\documentclass{beamer}

\usepackage{beamerthemesplit}

\title{Ruby Reflection}
\author{Ryan Mulligan}
\date{\today}

\begin{document}

\frame{\titlepage}

\frame{
	\frametitle{What is metaprogramming?}
  
    Wikipedia: "Metaprogramming is the writing of computer programs that write or manipulate other programs (or themselves) as their data"
}

\frame{
	\frametitle{What is reflection?}
  
    Wikipedia: "The process by which a computer program can observe and modify its own structure and behavior at runtime."
}

\frame{
	\frametitle{Big picture}
  
    The full power of Ruby is available at almost any context.
}

\frame{
	\frametitle{Basics}

	\begin{description}
	\item[self] Default Object
	\item[.class] Class of an object
	\item[.methods] Methods of an object
	\item[.instance\_variables]
	\item[.class\_variables]
	\item[Singleton Class] a = Array.new; def a.size; puts "hello world";end;a.size
	\end{description}
}

\frame{
	\frametitle{Evals}

	\begin{description}
	\item[eval] Interpret a string of Ruby code
	\item[instance\_eval] Changes self for a block
	\item[class\_eval] equivalent to class C; block...; end;
	\end{description}

}

\frame{
	\frametitle{Gets/Sets/Sends}

	\begin{description}
	\item[instance\_variable\_get/set] a.instance\_variable\_set :@variable, value
	\item[class\_variable\_get/set]
	\item[const\_get/set] Array.const\_set :A, "hallow" Array::A
	\item[send] a.send :method\_name, args
	\end{description}
}

\frame{
	\frametitle{Methods}

	\begin{description}
	\item[Class\#remove\_method]
	\item[undef] opposite of def
	\item[method\_missing] Catchall
	\end{description}
}

\frame{
	\frametitle{BasicObject}

    New in 1.9. BasicObjet: an Object with no methods, class, or instance variables.
}

\frame {\frametitle{Exercises}

       http://ruby-metaprogramming.rubylearning.com/html/
}

\frame {\frametitle{Questions} Questions}

\end{document}

